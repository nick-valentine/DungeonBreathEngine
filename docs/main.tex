%%%%%%%%%%%%%%%%%%%%%%%%%%%%%%%%%%%%%%%%%
% The Legrand Orange Book
% LaTeX Template
% Version 2.3 (8/8/17)
%
% This template has been downloaded from:
% http://www.LaTeXTemplates.com
%
% Original author:
% Mathias Legrand (legrand.mathias@gmail.com) with modifications by:
% Vel (vel@latextemplates.com)
%
% License:
% CC BY-NC-SA 3.0 (http://creativecommons.org/licenses/by-nc-sa/3.0/)
%
% Compiling this template:
% This template uses biber for its bibliography and makeindex for its index.
% When you first open the template, compile it from the command line with the 
% commands below to make sure your LaTeX distribution is configured correctly:
%
% 1) pdflatex main
% 2) makeindex main.idx -s StyleInd.ist
% 3) biber main
% 4) pdflatex main x 2
%
% After this, when you wish to update the bibliography/index use the appropriate
% command above and make sure to compile with pdflatex several times 
% afterwards to propagate your changes to the document.
%
% This template also uses a number of packages which may need to be
% updated to the newest versions for the template to compile. It is strongly
% recommended you update your LaTeX distribution if you have any
% compilation errors.
%
% Important note:
% Chapter heading images should have a 2:1 width:height ratio,
% e.g. 920px width and 460px height.
%
%%%%%%%%%%%%%%%%%%%%%%%%%%%%%%%%%%%%%%%%%

%----------------------------------------------------------------------------------------
%	PACKAGES AND OTHER DOCUMENT CONFIGURATIONS
%----------------------------------------------------------------------------------------

\documentclass[11pt,fleqn]{book} % Default font size and left-justified equations

%----------------------------------------------------------------------------------------

\input{structure} % Insert the commands.tex file which contains the majority of the structure behind the template

\begin{document}

%----------------------------------------------------------------------------------------
%	TITLE PAGE
%----------------------------------------------------------------------------------------

\begingroup
\thispagestyle{empty}
\begin{tikzpicture}[remember picture,overlay]
\node[inner sep=0pt] (background) at (current page.center) {\includegraphics[width=\paperwidth]{background}};
\draw (current page.center) node [fill=ocre!30!white,fill opacity=0.6,text opacity=1,inner sep=1cm]{\Huge\centering\bfseries\sffamily\parbox[c][][t]{\paperwidth}{\centering DungeonBreath Manual\\[15pt] % Book title
{\Large Making a Game}\\[20pt] % Subtitle
{\huge Nick Valentine}}}; % Author name
\end{tikzpicture}
\vfill
\endgroup

%----------------------------------------------------------------------------------------
%	COPYRIGHT PAGE
%----------------------------------------------------------------------------------------

% \newpage
% ~\vfill
% \thispagestyle{empty}
% 
% \noindent Copyright \copyright\ 2018 John Smith\\ % Copyright notice
% 
% \noindent \textsc{Published by Publisher}\\ % Publisher
% 
% \noindent \textsc{book-website.com}\\ % URL
% 
% \noindent Licensed under the Creative Commons Attribution-NonCommercial 3.0 Unported License (the ``License''). You may not use this file except in compliance with the License. You may obtain a copy of the License at \url{http://creativecommons.org/licenses/by-nc/3.0}. Unless required by applicable law or agreed to in writing, software distributed under the License is distributed on an \textsc{``as is'' basis, without warranties or conditions of any kind}, either express or implied. See the License for the specific language governing permissions and limitations under the License.\\ % License information
% 
% \noindent \textit{First printing, March 2013} % Printing/edition date

%----------------------------------------------------------------------------------------
%	TABLE OF CONTENTS
%----------------------------------------------------------------------------------------

%\usechapterimagefalse % If you don't want to include a chapter image, use this to toggle images off - it can be enabled later with \usechapterimagetrue

\chapterimage{chapter_head_1.pdf} % Table of contents heading image

\pagestyle{empty} % No headers

\tableofcontents % Print the table of contents itself

\cleardoublepage % Forces the first chapter to start on an odd page so it's on the right

\pagestyle{fancy} % Print headers again

%----------------------------------------------------------------------------------------
%	PART
%----------------------------------------------------------------------------------------

\part{Part One}

%----------------------------------------------------------------------------------------
%	CHAPTER 1
%----------------------------------------------------------------------------------------
\chapterimage{chapter_head_2.pdf}

\chapter{General}

\section{Input}\index{Paragraphs of Text}

In order to interact with the game in any way, shape or form, you will need to pass input into it. These inputs are defined in
\begin{theorem}[Config file]
 GameDir/GameData/config.txt
\end{theorem}
This file is a simple key value store where the first word on any line is the key, and the second is the value. Input definitions follow a very
simple domain specific language in order to be defined.
\begin{theorem}[Input DSN]
device:[input number]:[axis or button]:[min value]:button
\end{theorem}
device may be either k for keyboard, or j for joystick. If the device is keyboard, then don't pass any of the optional values. The configuration would look something
like k:d. However if the input device is a joystick, all options are required as the input number is which joystick to look at (1-8)[hopefully this will be refactored out soon]
axis or button tells how to treat the input, and min value is useful in the case of axis to present a dead zone which will prevent false positive triggering.

\chapterimage{chapter_head_2.pdf} % Chapter heading image

\chapter{In Engine Creation}

\section{Tile Sets}

\subsection{Image layout}\index{Paragraphs of Text}

The engine for DungeonBreath allows for variable size sprites, and variable size tilesets. However it will tend to work best when all images are of a size
that is a power of 2, and no smaller than 16x16. Example sizes can be 32x32, 64x64, 256x256, and they don't have to be square, they can also be 32x128 if necessary.
In order to import your sprites into the engine, lay them out into one single sprite sheet per use case. A hero should have their own sheet, separate from other
NPCs. The most restrictive use case of separating sprites by use case is levels, each level may only have one tileset in use at a time.
Once the tilesheet is layed out properly, add it to:
\begin{theorem}[Image data directory]
GameDir/GameData/img/
\end{theorem}

\subsection{The tileset tool}
Then you may launch the game in dev mode and press the ``Tile Sets'' button. Choose New and provide an easy symbolic name, the name of the spritesheet with none of the 
path before the img directory (images may be placed inside of sub folders). And provide as the base size, the smallest unit of measure necessary to accurately select
the smallest sprite.

Once finished press next and you will be in the tileset edit screen, accept will commit the accumulated tiles up to this point as a tileset ending with the tile selected
by hitting accept. If there are no previously accumulated tiles, this will become a static tile. Static tiles do not have animations and will get some engine optimizations.
In order to accumulate tiles for animations, use fire while hovering over the tile you wish to select. Use this method to select all except for the very last tile, use enter 
on the last tile to commit as a single tileset.

\section{Level Creation}
Launch the game in dev mode and press the ``Level Editor'' button. You may choose new, or from a list of existing levels. Once in the level editor:
\begin{enumerate}
 \item accept + fire will remove a tile
 \item accept + alt\_fire will enter actor select mode
 \item alt\_fire will enter tile select mode
 \item fire will place the currently selected item at the cursor's position
 \item escape will enter the escape menu where the user may set the layer, collision type, save, exit, or return to editing
 \item up will move the cursor up one block
 \item down will move the cursor down one block
 \item left will move the cursor left one block
 \item right will move the cursor right one block
 \item next will place a collision cube at the cursor
\end{enumerate}
At this time in order to choose what events are triggered by collisions, you must manually edit the world file and add to that section. The dsn is:
\begin{theorem}[action dsn]
 <type> <action> <target>
\end{theorem}
Where type is the integer associated with the collision type, action is either teleport or call, and target is either the new world to load, or the lua function to call.


\chapter{Scripting}
\section{API}
\subsection{Audio}
Functions:

\begin{equation}
music.play(<string> name) \rightarrow <void>
\end{equation}
play a song located in "./GameData/sound/" with the filename name


\begin{equation}
music.stop() \rightarrow <void>
\end{equation}
stop all currently playing music

\begin{equation}
music.playing() \rightarrow <string>
\end{equation}
get the name of the currently playing song

\begin{equation}
music.set\_volume(<float> volume) \rightarrow <void>
\end{equation}
set the volume

\subsection{Core}
\subsubsection{config}
\begin{equation}
config.get\_int(<string> name, <int> default) \rightarrow <int>
\end{equation}
get an int configuration option, provide a default. if config.save() is called, and the default was used, the default will be persisted.

\begin{equation}
config.get\_sring(<string> name, <string> default) \rightarrow <string>
\end{equation}
get an string configuration option, provide a default. if config.save() is called, and the default was used, the default will be persisted.

\begin{equation}
config.set(<string> name, <string|int> value) \rightarrow <void>
\end{equation}
change a configuration option. if config.save() is called, the change will be persisted.

\begin{equation}
config.save() \rightarrow <void>
\end{equation}
persist all configuration changes.

\subsubsection{input}
\begin{enumerate}
 \item input.up
 \item input.down
 \item input.left
 \item input.right
 \item input.escape
 \item input.accept
 \item input.fire
 \item input.alt\_fire
 \item input.num\_keys
\end{enumerate}
Button enumerations.

\begin{equation}
input.is\_key\_pressed() \rightarrow <num>
\end{equation}
check if a button is pressed.

\subsubsection{lang}
\begin{equation}
lang.next() \rightarrow <string>
\end{equation}
change the current language to the next one in the index. returns the language name.

\begin{equation}
lang.prev() \rightarrow <string>
\end{equation}
change the current language to the previous one in the index. returns the language name.

\subsubsection{logger}
\begin{equation}
logger.debug(<string> value, ...) \rightarrow <void>
\end{equation}
log at the debug level.

\begin{equation}
logger.info(<string> value, ...) \rightarrow <void>
\end{equation}
log at the info level.

\begin{equation}
logger.warn(<string> value, ...) \rightarrow <void>
\end{equation}
log at the warn level.

\begin{equation}
logger.error(<string> value, ...) \rightarrow <void>
\end{equation}
log at the error level.

\subsubsection{index}
\begin{equation}
index.get(<string> path) \rightarrow <Index>
\end{equation}
get an Index object that references path, all indexes got must be released.

\begin{equation}
index.release(<Index> idx) \rightarrow <void>
\end{equation}
release a Index object so it can be removed from memory.

\begin{equation}
index.add(<Index> idx, <string> value) \rightarrow <void>
\end{equation}
add a value to this index, changes must be saved in order to be reflected onto disk.

\begin{equation}
index.remove(<Index> idx, <string> value) \rightarrow <void>
\end{equation}
remove a value from this index, changes must be saved in order to be reflected onto disk.

\begin{equation}
index.save(<Index> idx) \rightarrow <void>
\end{equation}
save all changes made to this index.

\begin{equation}
index.all(<Index> idx) \rightarrow <[]string>
\end{equation}
get all items in this index.

\subsubsection{imgui}
This library is a dev tool for ui's that normal users will not see.

\begin{equation}
imgui.start(<string> name) \rightarrow <void>
\end{equation}
create a new imgui window.

\begin{equation}
imgui.stop() \rightarrow <void>
\end{equation}
end the current imgui window.

\begin{equation}
imgui.start\_child(<string> name, <vec2> pos) \rightarrow <void>
\end{equation}
create a new imgui sub window.

\begin{equation}
imgui.stop\_child() \rightarrow <void>
\end{equation}
end a imgui sub window.

\begin{equation}
imgui.button(<string> name) \rightarrow <bool>
\end{equation}
create a new imgui button.

\begin{equation}
imgui.checkbox(<string> name, <bool> checked) \rightarrow <bool>
\end{equation}
create a new imgui checkbox.

\begin{equation}
imgui.progressbar(<float> value) \rightarrow <void>
\end{equation}
create a new imgui progressbar.

\begin{equation}
imgui.text(<string> value) \rightarrow <void>
\end{equation}
create a new imgui text.

\begin{equation}
imgui.input\_text(<string> label, <string> value) \rightarrow <string>
\end{equation}
create a new imgui text input.

\begin{equation}
imgui.input\_int(<string> label, <int> value) \rightarrow <int>
\end{equation}
create a new imgui int input.

\begin{equation}
imgui.input\_float(<string> label, <float> value) \rightarrow <float>
\end{equation}
create a new imgui float input.

\begin{equation}
imgui.image\_button(<Sprite> image) \rightarrow <bool>
\end{equation}
create a new imgui image button.

\begin{equation}
imgui.listbox(<string> label, <int> current, <[]string> values) \rightarrow <int>
\end{equation}
create a new imgui listbox, returns currently selected.

\subsection{Play}
\subsubsection{world}

\begin{equation}
world.change\_level(<World> w, <string> name, <vec2> player\_location) \rightarrow <void>
\end{equation}
request a new level load.

\begin{equation}
world.get\_actorman(<World> w) \rightarrow <ActorManager>
\end{equation}
get the world's actormanager.

\begin{equation}
world.draw(<World> w, <RenderWindow> win) \rightarrow <void>
\end{equation}
draw the world.

\begin{equation}
world.draw\_layer(<World> w, <RenderWindow> win, <int> layer) \rightarrow <void>
\end{equation}
draw the a layer of world.

\begin{equation}
world.draw\_actors(<World> w, <RenderWindow> win) \rightarrow <void>
\end{equation}
draw the actors in this world.

\begin{equation}
world.set\_edit\_mode(<World> w, <bool> e) \rightarrow <void>
\end{equation}
change the mode the world is in to be more usable for editing the world.

\begin{equation}
world.save\_edits(<World> w) \rightarrow <void>
\end{equation}
save all changes made to this world back onto the original world file.

\begin{equation}
world.add\_actor(<World> w, <string> name, <vec2> pos) \rightarrow <void>
\end{equation}
add an actor into this world.

\begin{equation}
world.add\_collision(<World> w, <int> layer, <vec2> pos) \rightarrow <void>
\end{equation}
add a collision into this world.

\begin{equation}
world.set\_tile(<World> w, <Tile> tile, <int> layer, <vec2> pos) \rightarrow <void>
\end{equation}
change a tile in this world.

\begin{equation}
world.remove\_tile(<World> w, <int> layer, <vec2> pos) \rightarrow <void>
\end{equation}
remove a tile from this world.

\begin{equation}
world.get\_size(<World> w) \rightarrow <vec2>
\end{equation}
get the size in tiles of this world.


\begin{equation}
world.get\_tileset(<World> w) \rightarrow <string>
\end{equation}
get the name of the tileset of this world.

\begin{equation}
world.get\_script\_name(<World> w) \rightarrow <string>
\end{equation}
get the name of the script of this world.

\subsubsection{Actor}
\begin{equation}
actor.get\_rect(<Actor> a) \rightarrow <rect>
\end{equation}
get the actor's collision rect

\begin{equation}
actor.set\_scale(<Actor> a, <vec2> v) \rightarrow <void>
\end{equation}
set the actor's scale

\begin{equation}
actor.set\_origin(<Actor> a, <vec2> v) \rightarrow <void>
\end{equation}
set the actor's origin

\begin{equation}
actor.set\_collision\_bounds(<Actor> a, <vec2> size) \rightarrow <void>
\end{equation}
set the actor's collision bounds

\begin{equation}
actor.get\_velocity(<Actor> a) \rightarrow <vec2>
\end{equation}
get the actor's velocity

\begin{equation}
actor.set\_velocity(<Actor> a, <vec2> v) \rightarrow <void>
\end{equation}
set the actor's velocity

\begin{equation}
actor.set\_tileset(<Actor> a, <int> t) \rightarrow <void>
\end{equation}
set the actor's tileset

\begin{equation}
actor.pause\_anim(<Actor> a) \rightarrow <void>
\end{equation}
pause the actor's animation

\begin{equation}
actor.play\_anim(<Actor> a) \rightarrow <void>
\end{equation}
play the actor's animation

\begin{equation}
actor.reset\_anim(<Actor> a) \rightarrow <void>
\end{equation}
reset the actor's animation

\begin{equation}
actor.draw(<Actor> a, <Window> w) \rightarrow <void>
\end{equation}
draw the actor

\subsubsection{Actor Manager}

\begin{equation}
actor\_manager.spawn(<ActorManager> a, <string> name, <vec2> pos) \rightarrow <actor\_handle>
\end{equation}
spawn an actor

\begin{equation}
actor\_manager.remove(<ActorManager> a, <int> actor\_handle) \rightarrow <void>
\end{equation}
remove an actor

\begin{equation}
actor\_manager.clear(<ActorManager> a) \rightarrow <void>
\end{equation}
clear out all actors

\begin{equation}
actor\_manager.set\_camera\_target(<ActorManager> a, <actor\_handle> h) \rightarrow <void>
\end{equation}
set the actor that the camera should follow

\begin{equation}
actor\_manager.get\_camera\_target(<ActorManager> a) \rightarrow <Actor>
\end{equation}
get the actor that the camera should follow

\begin{equation}
actor\_manager.get\_player(<ActorManager> a) \rightarrow <Actor>
\end{equation}
get the actor that represents the player

\begin{equation}
actor\_manager.set\_player(<ActorManager> a, <actor\_handle> h) \rightarrow <void>
\end{equation}
set the actor that represents the player

\subsection{Render}
\subsubsection{Sprite Manager}

\begin{equation}
sprite\_manager.get(<string> texture\_name, <int> sprite\_name) \rightarrow <SpriteManager>
\end{equation}
get a new sprite manager, release should be called on all object yeilded from this method

\begin{equation}
sprite\_manager.release(<SpriteManager> s) \rightarrow <SpriteManager>
\end{equation}
frees this object from memory

\begin{equation}
sprite\_manager.make\_sprite(<SpriteManager> s, <vec2> pos, <vec2> size) \rightarrow <Sprite>
\end{equation}
creates a sprite on this sprite manager

\begin{equation}
sprite\_manager.remove\_sprite(<SpriteManager> s, <Sprite> spr) \rightarrow <void>
\end{equation}
removes a sprite from this sprite manager

\subsubsection{Sprite}
\begin{equation}
sprite.set\_position(<Sprite> s, <vec2> pos) \rightarrow <void>
\end{equation}
sets the position of a sprite

\begin{equation}
sprite.set\_scale(<Sprite> s, <vec2> pos) \rightarrow <void>
\end{equation}
sets the scale of a sprite

\begin{equation}
sprite.draw(<Sprite> s, <Window> win) \rightarrow <void>
\end{equation}
draws a sprite to the window

\subsubsection{TileSet}
\begin{equation}
tile\_set.get(<string> tileset\_name) \rightarrow <TileSet>
\end{equation}
get a new tileset, release should be called on all object yeilded from this method

\begin{equation}
tile\_set.release(<TileSet> t) \rightarrow <void>
\end{equation}
frees this object from memory

\begin{equation}
tile\_set.keys(<TileSet> t) \rightarrow <[]int>
\end{equation}
get all of the possible keys for tiles that can be spawned

\begin{equation}
tile\_set.get\_tile(<TileSet> t, <int> key) \rightarrow <Tile>
\end{equation}
get a tile from this tileset, release should be called on all object yeilded from this method

\subsubsection{Tile}
\begin{equation}
tile.release(<Tile> t) \rightarrow <void>
\end{equation}
frees this object from memory

\begin{equation}
tile.update(<Tile> t, <int> delta) \rightarrow <void>
\end{equation}
update this tile

\begin{equation}
tile.draw(<Tile> t, <Window> window) \rightarrow <void>
\end{equation}
draw this tile to the window

\begin{equation}
tile.play(<Tile> t) \rightarrow <void>
\end{equation}
play this tile's animation.

\begin{equation}
tile.pause(<Tile> t) \rightarrow <void>
\end{equation}
pause this tile's animation.

\begin{equation}
tile.reset(<Tile> t) \rightarrow <void>
\end{equation}
reset this tile's animation.

\begin{equation}
tile.set\_location(<Tile> t, <vec2> pos) \rightarrow <void>
\end{equation}
set this tile's location

\begin{equation}
tile.get\_location(<Tile> t) \rightarrow <vec2>
\end{equation}
get this tile's location

\begin{equation}
tile.set\_scale(<Tile> t, <vec2> s) \rightarrow <void>
\end{equation}
set this tile's scale

\begin{equation}
tile.set\_origin(<Tile> t, <vec2> o) \rightarrow <void>
\end{equation}
set this tile's origin

\begin{equation}
tile.get\_icon(<Tile> t) \rightarrow <Sprite>
\end{equation}
set this tile's icon

\subsection{Scene}
\begin{equation}
scene.push(<Scene> s, <string> name) \rightarrow <void>
\end{equation}
push a new scene onto the scene stack

\begin{equation}
scene.pop(<Scene> s) \rightarrow <void>
\end{equation}
pop this scene from the scene stack

\begin{equation}
scene.get\_menu(<Scene> s) \rightarrow <Menu>
\end{equation}
get the menu owned by this scene

\begin{equation}
scene.get\_size(<Scene> s) \rightarrow <vec2>
\end{equation}
get the size of this scene

\begin{equation}
scene.reset\_camera(<Scene> s) \rightarrow <void>
\end{equation}
reset the scene's camera back to 0

\begin{equation}
scene.get\_camera\_center(<Scene> s) \rightarrow <vec2>
\end{equation}
get the camera's center

\begin{equation}
scene.move\_camera(<Scene> s, <vec2> diff) \rightarrow <void>
\end{equation}
move the camera by diff

\begin{equation}
scene.set\_viewport(<Scene> s, <rect> r) \rightarrow <void>
\end{equation}
set the view to r

\begin{equation}
scene.get\_viewport(<Scene> s) \rightarrow <rect>
\end{equation}
get the scene's viewport

\begin{equation}
scene.zoom\_camera(<Scene> s, <float> mult) \rightarrow <void>
\end{equation}
zoom the scene by mult

\begin{equation}
scene.apply\_view(<Scene> s, <Window> window) \rightarrow <void>
\end{equation}
apply the scene's camera

\begin{equation}
scene.init\_world(<Scene> s) \rightarrow <void>
\end{equation}
initialize the scene's world

\begin{equation}
scene.get\_world(<Scene> s) \rightarrow <World>
\end{equation}
get the scene's world

\begin{equation}
scene.draw(<Scene> s, <Window> win) \rightarrow <void>
\end{equation}
draw the scene

\subsection{Shape}
\subsubsection{Rectangle Shape}

\begin{equation}
rectangle\_shape.get() \rightarrow <RectangleShape>
\end{equation}
get a rectangle, release should be called on all object yeilded from this method

\begin{equation}
rectangle\_shape.release(<RectangleShape> r) \rightarrow <void>
\end{equation}
free this object from memory

\begin{equation}
rectangle\_shape.set\_size(<RectangleShape> r, <vec2> s) \rightarrow <void>
\end{equation}
set this objects size

\begin{equation}
rectangle\_shape.set\_position(<RectangleShape> r, <vec2> p) \rightarrow <void>
\end{equation}
set this objects position

\begin{equation}
rectangle\_shape.set\_fill\_color(<RectangleShape> r, <rgb[a]> p) \rightarrow <void>
\end{equation}
set this objects fill color

\begin{equation}
rectangle\_shape.set\_outline\_color(<RectangleShape> r, <rgb[a]> p) \rightarrow <void>
\end{equation}
set this objects outline color

\begin{equation}
rectangle\_shape.set\_outline\_thickness(<RectangleShape> r, <float> s) \rightarrow <void>
\end{equation}
set this objects outline thickness

\begin{equation}
rectangle\_shape.draw(<RectangleShape> r, <Window> win) \rightarrow <void>
\end{equation}
draw this object to window

\subsection{UI}
\subsubsection{Menu}
\begin{equation}
menu.set\_current(<Menu> m, <MenuItem> mi) \rightarrow <void>
\end{equation}
set the currently hovered menu item

\begin{equation}
menu.add\_sprite\_button(<Menu> m, <string> tag, <rect> pos, <string> tex, <rect> normal, <rect> hover, <rect> press) \rightarrow <MenuItem>
\end{equation}
insert a spritebutton into this menu

\begin{equation}
menu.add\_label(<Menu> m, <rect> pos, <string> content\_key) \rightarrow <MenuItem>
\end{equation}
insert a label into this menu, content\_key is a lookup into the active lang file.

\begin{equation}
menu.add\_label\_raw(<Menu> m, <rect> pos, <string> content) \rightarrow <MenuItem>
\end{equation}
insert a label into this menu

\begin{equation}
menu.has\_signal(<Menu> m) \rightarrow <bool>
\end{equation}
see if this menu has a signal to be processed

\begin{equation}
menu.signal\_tag(<Menu> m) \rightarrow <string>
\end{equation}
get the signals ui tag

\begin{equation}
menu.signal\_str(<Menu> m) \rightarrow <string>
\end{equation}
get the signals string message

\begin{equation}
menu.signal\_int(<Menu> m) \rightarrow <int>
\end{equation}
get the signals int message

\begin{equation}
menu.clear(<Menu> m) \rightarrow <void>
\end{equation}
empty out this menu

\subsubsection{MenuItem}
\begin{equation}
menu\_item.raw(<MenuItem> m) \rightarrow <MenuItem>
\end{equation}
get the underlying object in this menu item

\begin{equation}
menu\_item.set\_up(<MenuItem> m, <MenuItem> o) \rightarrow <void>
\end{equation}
set the relationship between two buttons

\begin{equation}
menu\_item.set\_down(<MenuItem> m, <MenuItem> o) \rightarrow <void>
\end{equation}
set the relationship between two buttons

\begin{equation}
menu\_item.set\_left(<MenuItem> m, <MenuItem> o) \rightarrow <void>
\end{equation}
set the relationship between two buttons

\begin{equation}
menu\_item.set\_right(<MenuItem> m, <MenuItem> o) \rightarrow <void>
\end{equation}
set the relationship between two buttons

\begin{equation}
menu\_item.pair\_items(<MenuItem> m, <MenuItem> o, <int> direction) \rightarrow <void>
\end{equation}
set the relationship between two buttons

\begin{equation}
menu\_item.get\_tag(<MenuItem> m) \rightarrow <string>
\end{equation}
get the tag set when creating this item

\subsubsection{Label}

\begin{equation}
label.get(<rect> p, <string> contents) \rightarrow <Label>
\end{equation}
create a new unmanaged label, all labels created by this method should be released

\begin{equation}
label.release(<Label> l) \rightarrow <void>
\end{equation}
release an unmanaged label

\begin{equation}
label.set\_position(<Label> l, <rect> pos) \rightarrow <void>
\end{equation}
set this objects position

\begin{equation}
label.set\_string(<Label> l, <string> contents) \rightarrow <void>
\end{equation}
set this objects string

\begin{equation}
label.set\_size(<Label> l, <float> size) \rightarrow <void>
\end{equation}
set this objects char size

\begin{equation}
label.draw(<Label> l, <Window> win) \rightarrow <void>
\end{equation}
draw this object

\section{Hooks}
There are three main script hooks into this engine.
\subsection{Actors}
An actor is anything that moves while on screen, actors are composed of a global table named 'me' which is to contain a tileset string variable denoting which tileset the actor wishes to use. An update function of the form:
\begin{equation}
me.update(<int> delta) \rightarrow <void>
\end{equation}
an init function:
\begin{equation}
me.init() \rightarrow <void>
\end{equation}
a draw function, which is optional in order to override default drawing behavior:
\begin{equation}
me.draw(<Window> win) \rightarrow <void>
\end{equation}
and a release function to release resources when the actor is destructed
\begin{equation}
me.release() \rightarrow <void>
\end{equation}

\subsection{Levels}
Levels are scriptable add potential events to special collisions, in a level you can define one script to run, then in the collision section, map collision types to functions inside a me table inside that script. the functions should take no parameters and have no return. Like actor: draw is overridable.

\subsection{Scenes}
Scenes are your game state. Everything is a scene, menus, submenus, even the game itsself. scenes should export a me table as well, the me table should have the functions:
\begin{equation}
me.update(<int> delta) \rightarrow <void>
\end{equation}
an init function:
\begin{equation}
me.init() \rightarrow <void>
\end{equation}
a draw function, which is optional in order to override default drawing behavior:
\begin{equation}
me.draw(<Window> win) \rightarrow <void>
\end{equation}
a release function to release resources when the actor is destructed
\begin{equation}
me.release() \rightarrow <void>
\end{equation}
a wakeup function for when this scene comes back into focus
\begin{equation}
me.wakeup() \rightarrow <void>
\end{equation}
and a sleep function for when this scene goes out of focus
\begin{equation}
me.sleep() \rightarrow <void>
\end{equation}


%------------------------------------------------

%Tilesets\section{Citation}\index{Citation}
%
%This statement requires citation \cite{article_key}; this one is more specific \cite[162]{book_key}.
%
%%------------------------------------------------
%
%\section{Lists}\index{Lists}
%
%Lists are useful to present information in a concise and/or ordered way\footnote{Footnote example...}.
%
%\subsection{Numbered List}\index{Lists!Numbered List}
%
%\begin{enumerate}
%\item The first item
%\item The second item
%\item The third item
%\end{enumerate}
%
%\subsection{Bullet Points}\index{Lists!Bullet Points}
%
%\begin{itemize}
%\item The first item
%\item The second item
%\item The third item
%\end{itemize}
%
%\subsection{Descriptions and Definitions}\index{Lists!Descriptions and Definitions}
%
%\begin{description}
%\item[Name] Description
%\item[Word] Definition
%\item[Comment] Elaboration
%\end{description}
%
%%----------------------------------------------------------------------------------------
%%	CHAPTER 2
%%----------------------------------------------------------------------------------------
%
%\chapter{In-text Elements}
%
%\section{Theorems}\index{Theorems}
%
%This is an example of theorems.
%
%\subsection{Several equations}\index{Theorems!Several Equations}
%This is a theorem consisting of several equations.
%
%\begin{theorem}[Name of the theorem]
%In $E=\mathbb{R}^n$ all norms are equivalent. It has the properties:
%\begin{align}
%& \big| ||\mathbf{x}|| - ||\mathbf{y}|| \big|\leq || \mathbf{x}- \mathbf{y}||\\
%&  ||\sum_{i=1}^n\mathbf{x}_i||\leq \sum_{i=1}^n||\mathbf{x}_i||\quad\text{where $n$ is a finite integer}
%\end{align}
%\end{theorem}
%
%\subsection{Single Line}\index{Theorems!Single Line}
%This is a theorem consisting of just one line.
%
%\begin{theorem}
%A set $\mathcal{D}(G)$ in dense in $L^2(G)$, $|\cdot|_0$. 
%\end{theorem}
%
%%------------------------------------------------
%
%\section{Definitions}\index{Definitions}
%
%This is an example of a definition. A definition could be mathematical or it could define a concept.
%
%\begin{definition}[Definition name]
%Given a vector space $E$, a norm on $E$ is an application, denoted $||\cdot||$, $E$ in $\mathbb{R}^+=[0,+\infty[$ such that:
%\begin{align}
%& ||\mathbf{x}||=0\ \Rightarrow\ \mathbf{x}=\mathbf{0}\\
%& ||\lambda \mathbf{x}||=|\lambda|\cdot ||\mathbf{x}||\\
%& ||\mathbf{x}+\mathbf{y}||\leq ||\mathbf{x}||+||\mathbf{y}||
%\end{align}
%\end{definition}
%
%%------------------------------------------------
%
%\section{Notations}\index{Notations}
%
%\begin{notation}
%Given an open subset $G$ of $\mathbb{R}^n$, the set of functions $\varphi$ are:
%\begin{enumerate}
%\item Bounded support $G$;
%\item Infinitely differentiable;
%\end{enumerate}
%a vector space is denoted by $\mathcal{D}(G)$. 
%\end{notation}
%
%%------------------------------------------------
%
%\section{Remarks}\index{Remarks}
%
%This is an example of a remark.
%
%\begin{remark}
%The concepts presented here are now in conventional employment in mathematics. Vector spaces are taken over the field $\mathbb{K}=\mathbb{R}$, however, established properties are easily extended to $\mathbb{K}=\mathbb{C}$.
%\end{remark}
%
%%------------------------------------------------
%
%\section{Corollaries}\index{Corollaries}
%
%This is an example of a corollary.
%
%\begin{corollary}[Corollary name]
%The concepts presented here are now in conventional employment in mathematics. Vector spaces are taken over the field $\mathbb{K}=\mathbb{R}$, however, established properties are easily extended to $\mathbb{K}=\mathbb{C}$.
%\end{corollary}
%
%%------------------------------------------------
%
%\section{Propositions}\index{Propositions}
%
%This is an example of propositions.
%
%\subsection{Several equations}\index{Propositions!Several Equations}
%
%\begin{proposition}[Proposition name]
%It has the properties:
%\begin{align}
%& \big| ||\mathbf{x}|| - ||\mathbf{y}|| \big|\leq || \mathbf{x}- \mathbf{y}||\\
%&  ||\sum_{i=1}^n\mathbf{x}_i||\leq \sum_{i=1}^n||\mathbf{x}_i||\quad\text{where $n$ is a finite integer}
%\end{align}
%\end{proposition}
%
%\subsection{Single Line}\index{Propositions!Single Line}
%
%\begin{proposition} 
%Let $f,g\in L^2(G)$; if $\forall \varphi\in\mathcal{D}(G)$, $(f,\varphi)_0=(g,\varphi)_0$ then $f = g$. 
%\end{proposition}
%
%%------------------------------------------------
%
%\section{Examples}\index{Examples}
%
%This is an example of examples.
%
%\subsection{Equation and Text}\index{Examples!Equation and Text}
%
%\begin{example}
%Let $G=\{x\in\mathbb{R}^2:|x|<3\}$ and denoted by: $x^0=(1,1)$; consider the function:
%\begin{equation}
%f(x)=\left\{\begin{aligned} & \mathrm{e}^{|x|} & & \text{si $|x-x^0|\leq 1/2$}\\
%& 0 & & \text{si $|x-x^0|> 1/2$}\end{aligned}\right.
%\end{equation}
%The function $f$ has bounded support, we can take $A=\{x\in\mathbb{R}^2:|x-x^0|\leq 1/2+\epsilon\}$ for all $\epsilon\in\intoo{0}{5/2-\sqrt{2}}$.
%\end{example}
%
%\subsection{Paragraph of Text}\index{Examples!Paragraph of Text}
%
%\begin{example}[Example name]
%\lipsum[2]
%\end{example}
%
%%------------------------------------------------
%
%\section{Exercises}\index{Exercises}
%
%This is an example of an exercise.
%
%\begin{exercise}
%This is a good place to ask a question to test learning progress or further cement ideas into students' minds.
%\end{exercise}
%
%%------------------------------------------------
%
%\section{Problems}\index{Problems}
%
%\begin{problem}
%What is the average airspeed velocity of an unladen swallow?
%\end{problem}
%
%%------------------------------------------------
%
%\section{Vocabulary}\index{Vocabulary}
%
%Define a word to improve a students' vocabulary.
%
%\begin{vocabulary}[Word]
%Definition of word.
%\end{vocabulary}
%
%%----------------------------------------------------------------------------------------
%%	PART
%%----------------------------------------------------------------------------------------
%
%\part{Part Two}
%
%%----------------------------------------------------------------------------------------
%%	CHAPTER 3
%%----------------------------------------------------------------------------------------
%
%\chapterimage{chapter_head_1.pdf} % Chapter heading image
%
%\chapter{Presenting Information}
%
%\section{Table}\index{Table}
%
%\begin{table}[h]
%\centering
%\begin{tabular}{l l l}
%\toprule
%\textbf{Treatments} & \textbf{Response 1} & \textbf{Response 2}\\
%\midrule
%Treatment 1 & 0.0003262 & 0.562 \\
%Treatment 2 & 0.0015681 & 0.910 \\
%Treatment 3 & 0.0009271 & 0.296 \\
%\bottomrule
%\end{tabular}
%\caption{Table caption}
%\end{table}
%
%%------------------------------------------------
%
%\section{Figure}\index{Figure}
%
%\begin{figure}[h]
%\centering\includegraphics[scale=0.5]{placeholder}
%\caption{Figure caption}
%\end{figure}
%
%%----------------------------------------------------------------------------------------
%%	BIBLIOGRAPHY
%%----------------------------------------------------------------------------------------
%
%\chapter*{Bibliography}
%\addcontentsline{toc}{chapter}{\textcolor{ocre}{Bibliography}}
%
%%------------------------------------------------
%
%\section*{Articles}
%\addcontentsline{toc}{section}{Articles}
%\printbibliography[heading=bibempty,type=article]
%
%%------------------------------------------------
%
%\section*{Books}
%\addcontentsline{toc}{section}{Books}
%\printbibliography[heading=bibempty,type=book]

%----------------------------------------------------------------------------------------
%	INDEX
%----------------------------------------------------------------------------------------

\cleardoublepage
\phantomsection
\setlength{\columnsep}{0.75cm}
\addcontentsline{toc}{chapter}{\textcolor{ocre}{Index}}
\printindex

%----------------------------------------------------------------------------------------

\end{document}
